%http://www.informatik.uni-freiburg.de/~frank/ENG/latex-course/latex-course-3/latex-course-3_en.html
\documentclass{beamer}

\usepackage{graphicx}
\usepackage{textpos}
\usepackage{amsmath}
\usepackage{bm}
\usepackage{color} % For my tc command
\usepackage[labelformat=empty]{caption}
%\usepackage{algorithmic} % Need to install texlive
\def\wl{\par \vspace{\baselineskip}}
\def\imp{\Rightarrow}
\newcommand\tc[1]{\textcolor{red}{\textbf{#1}}}

% See this for more themes and colors: http://www.hartwork.org/beamer-theme-matrix/
\usepackage{beamerthemeHannover} % Determines the Theme
\usecolortheme{seahorse}         % Determines the Color

\title{Tornado - Random Forests}
%\logo{\includegraphics[width=1cm,height=1cm,keepspectration]{logo.png}}

\author[Arthur Lui]{Arthur L. Lui}
\institute[Brigham Young University]{
  Department of Statistics\\
  Brigham Young University
}
%Setting %%%%%%%%%%%%%%%%%%%%%%%%%%%%%%%%%%%%%%%%%%%%%%%%%%%%%%%%%%%%%%%%%%%%%%%%%

%- 5pts Problem Statement and Understanding 
%    Does the report summary sufficiently describe the background of the problem?
%    Are the goals of the analysis clearly stated?
%
%- 15pts Describe the method/model(s) that are used.
%    Was a brief description of method/model used given in the report?
%    Were any greek letters used clearly defined?  
%    Were any explicit or implicit assumptions needed to use the model adequately 
%    explained? (Collinearity, Linearity, Independence)
%
%- 10pts Model Justification
%    Does the report give reasons for why the particular model was chosen?
%    Does the report describe why this model is appropriate for this data and how 
%    it solves the current problem?
%    Are the assumptions of the model justified (e.g. via exploratory analysis)?
%
%- 15pts Results
%    Does the report adequately answer the questions posed in the case study?
%    Were estimates of the parameters and their uncertainties given?
%    Were the parameters interpreted in the context of the problem?
%    Did the report summarize the main points of the results in non-statistical 
%    terms?
%
%- 5pts Conclusions
%
%    Did the report discuss other potential approaches to solving the problem?
%    Did the report discuss any shortcomings of the approach/model used?
%    Did the report provide suggestions for next steps in the analysis or further 
%    questions that may be of interest?

\begin{document}

  \frame{\titlepage}

  \section{Introduction} % Sorah
    \frame{
      \frametitle{Introduction}
    }

  \section{Data} % Sorah
    \frame{
      \frametitle{Data}
        Show something about repeated observations
        How many observations for each Fscale
    }

    \frame{ % Sorah
      \frametitle{Data Cleaning}
    }
  
  \section{Model: Random Forests} % Sorah
    \frame{
      \frametitle{Model}
        Advantages:
                    p. 315,
                    Good for nonlinear, 
                    prediction, 
                    decreases variance coz less correlated
    }

    \frame{
      \frametitle{Random Forests}
        - Overview how to fit model (Binary Partitioning)
        - How we chose m (cv)
    }

    \frame{
      \frametitle{Model Assumptions}
        - Not Many
    }
  

  \section{Results}
    \frame{
      \frametitle{Results}
        - Show your results? (But how?)
        - How would a scientist use our model? (Using predictors and find Prediction)
        - Plot while holding other variables constant, and changing Loss,
          how the predicted Fscale changes.
        - Include a picture of a forest  
        - Include a tree. Using our model and a tree. 
            (2 tress: Width and Loss, Width Length)
        - Include Variable Importance    
        - Show confusion matrix. Point out how Fscale classification error 
          increases with Fscale.
    }

  \section{Conclusions}
    \frame{
      \frametitle{Conclusions}
        - black box. predicts well? 70\%
        - Objective, but uses past data.
    }

  \section{Future}
    \frame{
      \frametitle{Future}
        - Need more data about Fscale = 4.
        - Compare different methods. 
    }

  \section{Teamwork}
    \frame{
      \frametitle{Teamwork}
    }
\end{document}
