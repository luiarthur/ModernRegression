\documentclass{article}                                                   %
\usepackage{fullpage}                                                     %
\usepackage{pgffor}                                                       %
\usepackage{amssymb}                                                      %
\usepackage{Sweave}                                                       %
\usepackage{bm}                                                           %
\usepackage{mathtools}                                                    %
\usepackage{verbatim}                                                     %
\usepackage{appendix}                                                     %
\usepackage[UKenglish]{isodate} % for: \today                             %
\cleanlookdateon                % for: \today                             %
                                                                          %
\def\wl{\par \vspace{\baselineskip}}                                      %
\def\beginmyfig{\begin{figure}[htbp]\begin{center}}                       %
\def\endmyfig{\end{center}\end{figure}}                                   %
                                                                          %
\begin{document}                                                          %
% my title:                                                               %
\begin{center}                                                            %
  \section*{\textbf{Stat536 Midterm - Ozone Data}}                        %
  \subsection*{\textbf{Arthur Lui}}                                       %
  \subsection*{\noindent\today}                                           %
\end{center}                                                              %
\setkeys{Gin}{width=1\textwidth}                                          %
%%%%%%%%%%%%%%%%%%%%%%%%%%%%%%%%%%%%%%%%%%%%%%%%%%%%%%%%%%%%%%%%%%%%%%%%%%%

\section{Introduction}
  Ozone (O$_3$), a gas in the atmosphere, protects humans from the sun's UV
  radiation. However, ozone that is close to the ground can be dangerous to
  humans. Ozone is formed when pollutants react with each other in the presence
  of heat. It is the main component of smog. Enhaling high concentrations of
  O$_3$ triggers chest pain, bronchitis, emphysema, asthma, etc. Scientists have
  monitored O$_3$ levels by 1) direct measurement at measurement stations and 2)
  mathematically simulating the measurements using Community Multi-scale Air
  Quality Model (CMAQ). CMAQ O$_3$ measurements are simulated (on a fine spatial
  scale) based on ground characteristics, temperature, urban density, etc.  So,
  CMAQ data is vast (see \textbf{Figure 1}), but not as accurate as direct
  measurements. However, direct measurements are sparse (see \textbf{Figure 2}).
  The Environmental Protection Agency (EPA), which monitors O$_3$ are,
  therefore, interested in understanding the relationship between CMAQ (which is
  inaccurate) and station measurements. They eventually hope to predict
  ground-level O$_3$ at many locations given CMAQ measurements and observed
  measurements. Using a data set provided by Dr.  Heaton, I will construct such
  a model using a Gaussian
  Process.

  \subsection{Data Exploration}
    \begin{figure}\begin{center}
      \caption{CMAQ measurements of O$_3$}
      \includegraphics{raw/cmaq.pdf}
    \end{center}\end{figure}

    \includegraphics{raw/ozone.pdf}

\section{Method \& Model}
  \subsection{Brief Description of method / models used}
  \subsection{Assumptions}
\section{Model Justification}
  \subsection{Why choose a Gaussian Process?}
  \subsection{How does the GP solve the problem?}
  \subsection{Are Assumptions Justified?}
    \includegraphics{raw/resids.pdf}
    \includegraphics{raw/qqnorm.pdf}
    \includegraphics{raw/hist.pdf}
\section{Results}
  \subsection{Estimates of Parameters and CI}
    % latex table generated in R 3.0.2 by xtable 1.7-1 package
% Tue Mar  4 16:51:56 2014
\begin{table}[ht]
\centering
\begin{tabular}{rrrr}
  \hline
 & Estimates & CI.Lo & CI.Hi \\ 
  \hline
beta0 & 7.93766 & 1.29157 & 14.58375 \\ 
  beta1 & -0.11039 & -0.26032 & 0.03954 \\ 
  beta2 & 0.20544 & 0.06538 & 0.34549 \\ 
  beta3 & 0.08400 & -0.04960 & 0.21761 \\ 
  beta4 & 0.20130 & 0.06727 & 0.33534 \\ 
  beta5 & -0.11577 & -0.24077 & 0.00924 \\ 
  beta6 & 0.17512 & 0.05527 & 0.29497 \\ 
  beta7 & 0.14480 & 0.01922 & 0.27037 \\ 
  beta8 & 0.09521 & -0.01989 & 0.21031 \\ 
  beta9 & 0.06374 & -0.06132 & 0.18880 \\ 
  beta10 & 0.04647 & -0.06892 & 0.16187 \\ 
   \hline
\end{tabular}
\end{table}

    \[
      \begin{pmatrix}
        \hat{\tau^2} \\
        \hat{\sigma^2} \\
        \hat{\phi}
      \end{pmatrix} =
      \begin{pmatrix} 19.61\\1.04\\40.99 \end{pmatrix}
    \]
  \subsection{Coverage \& MSE }
    % latex table generated in R 3.0.2 by xtable 1.7-1 package
% Tue Mar  4 16:34:29 2014
\begin{table}[ht]
\centering
\begin{tabular}{rrrr}
  \hline
 & Estimate & CI.Lower & CI.Upper \\ 
  \hline
Coverage & 0.931 & 0.913 & 0.949 \\ 
  MSE & 20979.047 & 11019.079 & 30939.014 \\ 
   \hline
\end{tabular}
\end{table}

  \subsection{Predictions \& Uncertainties}
    \includegraphics{raw/center.pdf}
    \includegraphics{raw/lower.pdf}
    \includegraphics{raw/upper.pdf}
  \subsection{Interpretation}
  \subsection{Summary of Main Points}
    \includegraphics{raw/all.pdf}
\section{Conclusion}
  \subsection{Potential Alternative Approaches}
  \subsection{Shortcomings of GP or the way covars were chosen}
  \subsection{Further Investigation}
\end{document}
